\documentclass{article}

\usepackage{fullpage}
%\usepackage{citesort}
%\usepackage{url}
%\usepackage{listings}
\usepackage{color}


% Commands for formatting figures
\newcommand{\mydefhead}[2]{\multicolumn{2}{l}{{#1}}&\mbox{\emph{#2}}\\}
\newcommand{\mydefcase}[2]{\qquad\qquad& #1 &\mbox{#2}\\}

% Commands for language expressions and values
\newcommand{\true}{\mbox{\tt true}}
\newcommand{\false}{\mbox{\tt false}}\title{Operational Semantics for Bool* Language}
\newcommand{\ife}[3]{\mbox{\tt if}~{#1}~\mbox{\tt then}~{#2}~\mbox{\tt else}~{#3}}
\newcommand{\suc}[1]{\mbox{\tt succ}~{#1}}
\newcommand{\prd}[1]{\mbox{\tt pred}~{#1}}
\newcommand{\secret}[1]{\mbox{\tt secret}~{#1}}

% Big-step evaluation relation.
% #1 is the expression.
% #2 is the resulting value.
\newcommand{\bstep}[2]{{#1} \Downarrow {#2}}
\newcommand{\bigstep}[4]{{#1} \Downarrow {#2} \quad {#3} \Downarrow {#4}}

% Command for formatting the name of the evaluation relation.
\newcommand{\rel}[1]{ \mbox{\sc [#1]} }

% Format for a big-step evaluation rule.
% #1 is the name of the rule.
% #2 are the premises.  Leave blank if there are none.
% #3 is the conclusion.
\newcommand{\bsrule}[3]{
  \rel{#1} &
  \frac{\strut\begin{array}{@{}c@{}} #2 \end{array}}
       {\strut\begin{array}{@{}c@{}} #3 \end{array}}
   \\~\\
}
\author{
  Thomas H. Austin \\
  Aarohi Chopra
  }
\begin{document}
\maketitle

Figure shows the list of expressions and values for the Bool* language. \\
 
Big-step evaluation relation:
\[
  \bstep{e}{v}
\]
The above line should be read as ``the expression $e$ evaluates to the value $v$''. \\
\begin{figure}
\caption{Big-step semantics for Bool*}
\label{fig:bigstep}
  {\bf Evaluation Rules:~~~ \fbox{$\bstep{e}{v}$}} \\
\[
\begin{array}{r@{\qquad\qquad}c}
\bsrule{B-Value}{}{
  \bstep{v}{v}
}
\bsrule{B-ExpPublic}{}{
\bstep{e}{v^L}
}
\bsrule{B-ExpSecret}{\bstep{e}{v^L}}{
\bstep{secret(e)}{v^H}
}
\bsrule{B-ifTruePublic}
{\bigstep{e_1^L}{True^L}{e_2^L}{v^L}
}{
\bstep{if \, e_1^L \, then \, e_2^L \, else \, e_3^L}{v^L}
}
\bsrule{B-ifTruePrivate}
{\bigstep{e_1^H}{True^H}{e_2^H}{v^H}
}{
\bstep{if \, e_1^H \, then \, e_2^H \, else \, e_3^H}{v^H}
}
\bsrule{B-ifFalsePublic}
{\bigstep{e_1^L}{False^L}{e_3^L}{v^L}
}{
\bstep{if \, e_1^L \, then \, e_2^L \, else \, e_3^L}{v^L}
}
\bsrule{B-ifFalsePrivate}
{\bigstep{e_1^H}{False^H}{e_3^H}{v^H}
}{
\bstep{if \, e_1^H \, then \, e_2^H \, else \, e_3^H}{v^H}
}
\bsrule{B-SuccPublic}
{\bstep{e^L}{n^L} \quad n'^L = n^L + 1
}{
\bstep{(succ e^L)}{n'^L}
}
\bsrule{B-SuccPrivate}
{\bstep{e^H}{n^H} \quad n'^H = n^H + 1
}{
\bstep{(succ e^H)}{n'^H}
}
\bsrule{B-PredPublic}
{\bstep{e^L}{n^L} \quad n'^L = n^L - 1
}{
\bstep{(pred e^L)}{n'^L}
}
\bsrule{B-PredPrivate}
{\bstep{e^H}{n^H} \quad n'^H = n^H - 1
}{
\bstep{(pred e^H)}{n'^H}
}
\end{array}
\]
\end{figure}

Figure~\ref{fig:bigstep} shows the big-step evaluation rules for the Bool* language.





\section*{Exercise}

Let's extend the language with new features.
Our new language will be called BoolNum* and add the secret function to it.

%\begin{figure}
%\caption{The BoolNum* language}
%\label{fig:lang2}
\[
  \begin{array}{ll@{\qquad\qquad}l}
  \mydefhead{e ::=\qquad\qquad\qquad\qquad}{Expressions}
  \mydefcase{\true}{true value}
  \mydefcase{\false}{false value}
  \mydefcase{n}{natural numbers}
  \mydefcase{\ife e e e}{conditional expressions}
  \mydefcase{\suc e}{successor}
  \mydefcase{\prd e}{predecessor}
  \mydefcase{\secret e}{Expression with a label}
  \\
  \mydefhead{v ::=\qquad\qquad\qquad\qquad}{Values}
  \mydefcase{\true}{true value}
  \mydefcase{\false}{false value}
  \mydefcase{n}{natural numbers}
\end{array}
\]
%\end{figure}



\end{document}

